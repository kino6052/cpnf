\documentclass[12pt]{article}
\usepackage{amsmath, amssymb, amsthm}
\usepackage{geometry}
\usepackage{hyperref}
\usepackage{cite}
\usepackage{algorithm}
\usepackage{algpseudocode}
\usepackage{titlesec}

\geometry{margin=1in}
\titleformat{\section}{\large\bfseries}{\thesection}{1em}{}
\titleformat{\subsection}{\normalsize\bfseries}{\thesubsection}{1em}{}

\title{A Computational Framework for Prime Numbers: \\ A Case Study on How Reification and Ontology Affects Prime Infinitude Arguments}
\author{Anonymous}
\date{}

\begin{document}

\maketitle

\begin{abstract}
This paper examines how \emph{reification}---the treatment of mathematical processes as completed objects---can introduce unnecessary obstacles into mathematical reasoning. We introduce the \textbf{Computational Prime Number Framework (CPNF)}, a formal model in which primality is treated not as an inherent property of integers but as the outcome of a stagewise certification process. Using CPNF as a case study, we show how shifting from a reified ontology of ``completed primes'' to a process-oriented ontology reframes questions of infinitude and pattern recurrence. Within this framework, we demonstrate that the twin-prime certification process cannot remain permanently stagnant---i.e., it cannot reach a state after which no new certifications ever occur. This result relies on elementary modular arithmetic and a fixed-modulus reduction that becomes natural within CPNF's stage-relative semantics. The paper does not claim to resolve classical conjectures, but rather illustrates how ontological reframing can alter the landscape of mathematical reasoning, revealing how reification may heighten perceived difficulty.

\textbf{Keywords:} Philosophy of mathematics, reification, computational number theory, prime numbers, ontology, constructive mathematics, process philosophy
\end{abstract}

\section*{Important Notice}
This paper presents a conceptual case study within a formally defined framework. It makes no claims about the truth or falsity of classical number-theoretic conjectures. The arguments offered are valid relative to the definitions, axioms, and procedures of CPNF.

\section*{Methodological Note}
The paper is primarily philosophical in aim: it investigates how reification---the conversion of dynamic processes into static objects---influences proof design and perceived difficulty. CPNF is offered as an explicit, self-contained model in which certain classical obstacles (e.g., the parity problem) do not arise because they are artifacts of reification. The mathematical development is intended to illustrate the philosophical point, not to serve as a complete formal proof in classical number theory.

\tableofcontents

\section{Introduction: The Problem of Reification in Mathematics}
Mathematical practice is often guided by implicit ontological commitments---assumptions about what mathematical objects \emph{are} and how they exist. One particularly pervasive commitment is \emph{reification}: the treatment of ongoing processes, dynamic constructions, or temporal verifications as if they were completed, static objects. This tendency, criticized by philosophers such as Henri Bergson \cite{bergson1911}, Alfred North Whitehead \cite{whitehead1929}, and more recently by process philosophers and some strands of constructive mathematics, can obscure the procedural nature of mathematical discovery and introduce artificial barriers in reasoning.

In number theory, primes are typically conceived as pre-existing objects with fixed properties. This reified view leads naturally to questions about the ``set of all primes'' or the ``distribution of twin primes'' as completed infinities---questions that are then approached with tools designed to measure or count such objects. But what if primes are more naturally understood as \emph{certifications} issued by an ongoing process of verification? This paper explores that alternative.

We introduce the \textbf{Computational Prime Number Framework (CPNF)}, a model that treats primality not as a static property but as a status conferred by a stepwise, verifiable filtering process. In CPNF, primes emerge as survivors of a deterministic sieve that proceeds in discrete stages, each stage certifying one new number as ``prime relative to current knowledge.'' This process-oriented ontology echoes constructive tendencies in Euclid \cite{euclid} and Eratosthenes, but also aligns with philosophical traditions that resist reification.

Our goal is not to prove new theorems in classical number theory, but to demonstrate how a shift away from reification can reshape mathematical reasoning. We focus on twin primes---pairs of the form $(6n-1, 6n+1)$---and show that within CPNF, the certification process for such pairs cannot enter permanent stagnation. The argument relies on elementary modular arithmetic and a \emph{fixed-modulus reduction}, which becomes natural within the stage-relative certification semantics of CPNF.

We proceed as follows:
\begin{itemize}
    \item \S 2 defines CPNF for ordinary primes and shows how a classical infinitude argument is re-expressed as a non-termination property.
    \item \S 3 extends CPNF to twin-prime patterns.
    \item \S 4 identifies the central conceptual hurdle---the \emph{endless chase} scenario---and resolves it via fixed-modulus reduction, explaining why adversarial covering is not modeled in CPNF.
    \item \S 5 reflects on how this approach circumvents classical obstacles like the parity problem by avoiding reification.
    \item \S 6 discusses the philosophical implications, engaging with contemporary work on ontology and process in mathematics.
\end{itemize}

This study suggests that certain long-standing barriers in number theory may be, in part, artifacts of reification---and that alternative, process-oriented perspectives can open up distinct, and sometimes simpler, forms of reasoning.

\section{CPNF: A Process-Oriented Ontology for Primality}
\subsection{Basic Framework}
Let $\mathbb{N}^+$ denote the positive integers. In CPNF, primality is not defined via divisibility alone, but as the outcome of a certification process. The core components are:
\begin{itemize}
    \item \textbf{Blocker set $B_k$}: A finite set of numbers certified by stage $k$. Initially $B_0 = \emptyset$.
    \item \textbf{Candidate set}: All integers $\geq 2$.
    \item \textbf{Sieve operator}:  
    \[
    S(B_k, \mathbb{N}^+) = \{ n \in \mathbb{N}^+ : \forall p \in B_k,\ p \nmid n \}.
    \]
\end{itemize}

The process proceeds in stages:
\begin{enumerate}
    \item \textbf{Initialization}: $B_0 = \emptyset$.
    \item \textbf{Stage $k$}:
    \begin{itemize}
        \item Compute survivors $ = S(B_{k-1}, \mathbb{N}^+)$.
        \item Select the minimal survivor $n$.
        \item Certify $n$ by adding it to $B_k$.
    \end{itemize}
\end{enumerate}

Thus, each certified number comes with a finite certificate: the list of blockers it survived.

\subsection{Properties of the Process}
\textbf{Soundness}: No composite number is ever certified. \\
\textit{Proof sketch}: If a composite $m$ were certified, its smallest prime divisor $p < m$ would have been certified earlier, contradicting $m$'s survival at its certification stage.

\textbf{Non-termination}: The process never halts. \\
\textit{Proof sketch}: If only finitely many numbers $p_1, \dots, p_m$ were certified, consider $M = p_1 \cdots p_m + 1$. It is not divisible by any $p_i$, so either $M$ or a smaller survivor would be selected next---a contradiction.

This reformulates a classical argument (Euclid \cite{euclid}), but here infinitude is expressed as the non-termination of a procedure, not as a statement about an infinite set.

\section{Extending CPNF to Twin-Prime Patterns}
\subsection{Twin-Index Representation}
After removing multiples of 2 and 3, all survivors > 3 have the form $6n \pm 1$. This suggests a compact representation:
\begin{itemize}
    \item A \textbf{twin index} $n$ corresponds to the pair $(6n-1, 6n+1)$.
    \item Certification of $n$ means that both numbers survive division by all blockers in the current set $B_k$.
\end{itemize}

\subsection{Certification Window}
To ensure correctness, a twin index $n$ is certified only when all potential divisors $\leq \sqrt{6n+1}$ are already in $B_k$. If $y_k = \max(B_k)$, this requires:
\[
\sqrt{6n+1} \leq y_k \quad \Rightarrow \quad n \leq W_k = \left\lfloor \frac{y_k^2 - 1}{6} \right\rfloor.
\]
$W_k$ is called the \textbf{certification window} at stage $k$.

\subsection{Twin Sifting Process}
The process alternates between:
\begin{enumerate}
    \item \textbf{Completion phase}: Ensure $B_k$ contains all survivors $\leq 6 \cdot \max(T) + 1$, where $T$ is the set of certified twin indices.
    \item \textbf{Search phase}: Look for the smallest $n \leq W_k$ not in $T$ such that both $6n-1$ and $6n+1$ survive division by all $p \in B_k$.
\end{enumerate}

If such an $n$ exists, certify it; otherwise, add the next ordinary survivor to $B_k$ and continue.

\section{The Endless Chase and Fixed-Modulus Reduction}
\subsection{The Endless Chase Scenario}
Could the process reach a state after which no new twin index is ever certified, even though $B_k$ keeps growing? This is the \textbf{endless chase}: the hypothetical scenario where all admissible twin indices forever lie outside the certification window.

\subsection{Why Density Heuristics Are Insufficient}
Classical density estimates suggest that the number of admissible indices in $[1, W_k]$ grows roughly like $y_k^2 / (\log y_k)^2$ (Hardy \& Littlewood \cite{hardy1923}). But because the modulus $M_k = \prod_{p \in B_k} p$ grows super-exponentially, $W_k$ is far shorter than a full period of the sieve lattice. Hence, average density does not guarantee existence in such a short interval.

\subsection{Fixed-Modulus Reduction as an Analytic Lens}
Here CPNF's stage-relative semantics enables a key simplification. Certification at stage $k$ depends only on blockers present at that stage. Therefore, when analyzing progress beyond a given stage $k_0$, we may \textbf{freeze} the modulus at $M_0 = \prod_{p \in B_{k_0}} p$ and work within the fixed periodic lattice of residues modulo $M_0$. This fixed-modulus reduction is an analytic device for reasoning about the process, not a modification of the CPNF procedure itself.

Each new prime $p > \max(B_{k_0})$ removes at most two residue classes modulo $p$, which translates to removing $M_0/p$ residues modulo $M_0$. Thus, the thinning of admissible residues is logarithmic in $y$, while the window $W(y) \sim y^2/6$ grows quadratically.

\subsection{Why Permanent Stagnation Cannot Occur}
Once $W(y)$ exceeds $M_0$, the window contains many full periods of the fixed lattice. The product $\prod (1 - 2/p)$ over primes $p > \max(B_{k_0})$ decays at most logarithmically (as $1/(\log y)^2$ up to a constant factor, by Mertens' third theorem \cite{mertens1874}), while $W(y)$ grows quadratically. Consequently, for sufficiently large $y$, the cumulative deletions grow too slowly to force exhaustion without a covering system, and the window---spanning many periods---can fail to intersect only if the induced deletions form a covering of $R_0$.

Crucially, in CPNF, we are free to choose any $k_0$ and corresponding $M_0$. The blockers are generated procedurally by the certification process, not selected adversarially. Therefore, the possibility of a ``conspiracy'' where deletions align to form a covering system modulo a chosen $M_0$ is excluded by construction: the process does not have the foresight or intent to engineer such a global cancellation. This aligns with the philosophical insight that reifying primes as a completed set becomes salient only under reified interpretations.

A more formal justification of this non-covering property is provided in Appendix \ref{app:D}.

Therefore, within CPNF, the twin certification process cannot remain permanently stagnant: it must eventually produce new certifications. The argument relies only on the growth rates and the periodic structure induced by the fixed modulus, not on uniform distribution or precise counting of survivors.

\section{Why This Shift Matters: Ontology, Reification, and Proof Strategy}
\subsection{Reification and the Parity Problem}
The parity problem---the inability of sieve methods to distinguish numbers with an even versus odd number of prime factors---is a classic example of an obstacle that arises from reification. It emerges when we treat the sieve as a measuring tool for a completed set of primes. In contrast, CPNF uses the sieve as a construction tool for an ongoing process. The parity problem becomes irrelevant because we are not trying to count or measure a reified set.

\begin{table}[h]
\centering
\begin{tabular}{|l|l|l|}
\hline
\textbf{Aspect} & \textbf{Classical Number Theory (Reified)} & \textbf{CPNF (Process-Oriented)} \\
\hline
\textbf{What is a prime?} & An integer $>1$ with exactly two divisors & A number certified at stage $k$ as surviving $B_k$ \\
\textbf{Goal} & Count/distribute true primes & Continue certification process indefinitely \\
\textbf{Parity relevance} & Fundamental obstacle to density estimation & Irrelevant---certifications are stage-relative \\
\hline
\end{tabular}
\caption{Comparison of classical vs. CPNF approaches}
\end{table}

\textbf{Why parity doesn't apply:}
\begin{enumerate}
    \item \textbf{No absolute primality claims}: CPNF certifications are explicitly stage-relative.
    \item \textbf{Process over properties}: CPNF studies the behavior of a certification machine, not the properties of integers.
    \item \textbf{Different success criteria}: Success in CPNF means the process never terminates or stagnates permanently.
\end{enumerate}

\subsection{The Dangers of Reification}
Reification, in the sense criticized by Bergson \cite{bergson1911} and Whitehead \cite{whitehead1929}, involves mistaking a dynamic, temporal process for a static, atemporal object. In mathematics, this can lead to:
\begin{itemize}
    \item \textbf{Unnecessary complexity}: Problems become framed in terms of completed infinities, requiring advanced analytic tools.
    \item \textbf{Hypothetical obstacles}: Like the adversarial covering scenarios excluded in CPNF.
    \item \textbf{Epistemic opacity}: The constructive origins of mathematical objects are obscured.
\end{itemize}

CPNF offers a way to resist reification by keeping the certification process at the forefront. This shift is not merely notational; it changes what counts as a valid proof and what obstacles are considered relevant.

\section{Discussion and Philosophical Reflections}
The CPNF case study illustrates how ontological choices---specifically, the choice between a reified and a process-oriented ontology---can reshape mathematical practice. By adopting a process-oriented ontology, we gain access to proof strategies that are structurally simpler and that avoid certain classical obstacles.

This does not mean that CPNF ``solves'' hard problems in classical number theory; rather, it shows that these problems can be reformulated in a framework that simplifies reasoning. This suggests that the difficulty of a problem may depend not only on the mathematical facts themselves, but also on the framework through which we approach them.

\subsection{Engagement with Contemporary Philosophy}
Recent work in philosophy of mathematics has explored similar themes. For example:
\begin{itemize}
    \item \textbf{Stewart Shapiro (1997)} on the ontology of mathematical structures \cite{shapiro1997}.
    \item \textbf{Otávio Bueno (2009)} on the role of representations in mathematical reasoning \cite{bueno2009}.
    \item \textbf{James Ladyman (2014)} on structural realism and process \cite{ladyman2014}.
    \item \textbf{John Burgess (2015)} on the reliability of mathematical inference \cite{burgess2015}.
\end{itemize}

CPNF aligns with constructive and process-oriented approaches that emphasize mathematical activity over static objects (see also Mancosu \cite{mancosu2008}, Ferreirós \cite{ferreiros2016}). It also resonates with \emph{modal} approaches to the infinite (e.g., Linnebo \cite{linnebo2013}) that treat infinite sets as potential rather than actual.

\subsection{Limitations and Future Directions}
CPNF is a deliberately simplified framework. Its philosophical value lies in its explicitness, not its direct applicability to classical number theory. Future work could explore:
\begin{itemize}
    \item How process-oriented ontologies might be applied in other mathematical domains (e.g., analysis, geometry).
    \item The relationship between CPNF and formal systems of constructive mathematics (e.g., Martin-Löf type theory \cite{martinlof1984}).
    \item The epistemic implications of treating mathematical objects as certifications rather than discoveries.
\end{itemize}

\section{Conclusion}
This paper has presented the Computational Prime Number Framework as a philosophical case study in the dangers of reification and the potential of process-oriented ontologies. By re-conceiving primes as certifications issued by a stagewise sieve, we reframed the infinitude of twin primes as a non-stagnation property of a deterministic process. The fixed-modulus reduction---a key technical step---becomes natural within CPNF's stage-relative semantics and allows an elementary resolution of the endless chase scenario, with adversarial covering excluded by construction.

We have shown that within a carefully defined process-oriented framework, the recurrence of twin-prime patterns can be established without encountering the parity barrier or relying on unproven distributional hypotheses. The broader suggestion is methodological: explicit reflection on ontological commitments---and resistance to reification---can reveal alternative pathways in mathematical reasoning. By exploring how different ways of ``being a mathematical object'' influence proof, we may enrich both mathematical practice and philosophical understanding.

The novelty of CPNF lies not in establishing new arithmetical facts, but in providing a formally explicit process-oriented ontology for primality, within which classical notions of infinitude, obstruction, and certification can be meaningfully disentangled and reanalyzed.

\appendix
\section{Formal Definitions and Algorithms}\label{app:A}
\subsection{The Computational Prime Number Framework (CPNF)}
\textbf{Definition A.1.1 (Blocker Set).} \\
For any stage $k \in \mathbb{N}$, the blocker set $B_k$ is a finite set of positive integers certified by the CPNF through stage $k$. Initially, $B_0 = \emptyset$.

\textbf{Definition A.1.2 (Sieve Operator).} \\
For any finite set $B \subset \mathbb{N}^+$ and any set $X \subseteq \mathbb{N}^+$, define:
\[
S(B, X) = \{ n \in X : \forall p \in B, \ p \nmid n \}.
\]

\begin{algorithm}
\caption{Ordinary Sifting Process}
\begin{algorithmic}[1]
\Function{ordinary\_sifting\_process}{}
    \State $B = \emptyset$
    \While{True}
        \State $survivors = S(B, \{n \geq 2\})$
        \If{$survivors = \emptyset$} \textbf{break} \EndIf
        \State $n = \min(survivors)$
        \State $B = B \cup \{n\}$
    \EndWhile
\EndFunction
\end{algorithmic}
\end{algorithm}

\textbf{Theorem A.1.4 (Soundness).} \\
No composite integer is ever selected by the ordinary CPNF process.

\textit{Proof.} Suppose a composite number $m$ were selected at some stage. Let $p$ be its smallest prime divisor. Since $p < m$ and $p$ divides $m$, $p$ would have been selected earlier (as the minimal survivor not divisible by any previous blocker). Then $p \in B$ would block $m$, contradicting $m$'s selection as a survivor. $\square$

\textbf{Theorem A.1.5 (Non-termination).} \\
The CPNF process for ordinary primes never terminates.

\textit{Proof.} Suppose only finitely many survivors $p_1, \dots, p_m$ were certified. Consider $M = p_1 p_2 \cdots p_m + 1$. This number is not divisible by any $p_i$, so $M \in S(\{p_1,\dots,p_m\}, \mathbb{N}^+)$. Since $M > \max(p_i)$, it (or a smaller candidate) would be selected as the next minimal survivor, contradicting the assumption of finiteness. This mirrors Euclid's classical proof \cite{euclid}. $\square$

\section{Twin Primes in CPNF}\label{app:B}
\subsection{Twin-Index Representation}
\textbf{Definition B.1.1 (Twin Index).} \\
For $n \geq 1$, the twin index $n$ corresponds to the pair $(6n-1, 6n+1)$.

\textbf{Definition B.1.2 (Admissibility).} \\
A twin index $n$ is admissible relative to blocker set $B$ if both $6n-1$ and $6n+1$ are not divisible by any $p \in B$.

\textbf{Definition B.1.3 (Certification Window).} \\
Given $y = \max(B)$, the certification window is:
\[
W(y) = \left\lfloor \frac{y^2 - 1}{6} \right\rfloor.
\]

\subsection{Twin Sifting Algorithm}
\begin{algorithm}
\caption{Twin Sifting Process}
\begin{algorithmic}[1]
\Function{twin\_sifting\_process}{}
    \State $B = \{2, 3\}$
    \State $T = \emptyset$ \Comment{certified twin indices}
    \While{True}
        \Comment{Completion phase}
        \While{$\max(B) < 6\cdot\max(T) + 1$}
            \State $B = B \cup \{\min(\text{ordinarySieve}(B))\}$
        \EndWhile
        \Comment{Twin search phase}
        \State $y = \max(B)$
        \State $W = \lfloor (y^2 - 1)/6 \rfloor$
        \State $candidates = \{n \leq W : n \notin T \land n \text{ admissible for } B\}$
        \If{$candidates \neq \emptyset$}
            \State $n = \min(candidates)$
            \State $T = T \cup \{n\}$
        \Else
            \State $B = B \cup \{\min(\text{ordinarySieve}(B))\}$
        \EndIf
    \EndWhile
\EndFunction
\end{algorithmic}
\end{algorithm}

\section{The Fixed-Modulus Reduction}\label{app:C}
\subsection{Mathematical Formulation}
\textbf{Definition C.1.1 (Stage-Relative Certification).} \\
A twin index $n$ is certified at stage $k$ if it is admissible relative to $B_k$ and $n \leq W(\max B_k)$.

\textbf{Lemma C.1.2 (Irreversibility).} \\
If $n$ is certified at stage $k$, it remains certified for all stages $k' > k$.

\textit{Proof.} Certification depends only on blockers present at the certification stage. Later additions cannot retroactively invalidate earlier certifications. $\square$

\subsection{The Fixed-Modulus Construction}
For any stage $k_0$, let:
\[
M_0 = \prod_{p \in B_{k_0}} p.
\]

\textbf{Definition C.2.1 (Initial Residue Set).}
\[
R_0 = \{ r \in [0, M_0-1] : \forall p \in B_{k_0},\ 6r \not\equiv \pm 1 \pmod{p} \}.
\]

\textbf{Lemma C.2.2 (Thinning Under Fixed Modulus).} \\
For any prime $p > \max B_{k_0}$, the condition $6n \not\equiv \pm 1 \pmod{p}$ eliminates at most $2M_0/p$ residues from $R_0$ when lifted modulo $M_0$.

\textit{Proof.} Each congruence condition modulo $p$ corresponds to removing a complete residue class modulo $M_0$ for each solution modulo $p$. This relies on the Chinese Remainder Theorem (Gauss \cite{gauss1801}). $\square$

\section{Structural Constraints on Permanent Stagnation}\label{app:D}
\subsection{What Permanent Stagnation Would Require}
The purpose of this appendix is \textbf{not} to prove that permanent stagnation is impossible in an absolute sense. Rather, it is to identify \textbf{precisely what mathematical structure would be required} for such stagnation to occur within CPNF.

We show that permanent stagnation would force the existence of a highly structured \textbf{covering system} relative to a fixed modulus. This relocates the obstruction from parity phenomena to a different, explicitly characterizable mechanism.

\subsection{Fixed-Modulus Reduction and Residue Sets}
Fix a stage $k_0$ and let
\[
M_0 = \prod_{p \in B_{k_0}} p.
\]

Let
\[
R_0 = \{ r \bmod M_0 : \forall p \in B_{k_0},\ 6r \not\equiv \pm 1 \pmod p \}.
\]

By construction, $R_0 \neq \varnothing$, since otherwise no twin index could have been certified prior to stage $k_0$.

For any later prime $q > \max(B_{k_0})$, the condition
\[
6n \not\equiv \pm 1 \pmod q
\]
removes at most two residue classes modulo $q$, which induces the removal of at most $2M_0/q$ residue classes modulo $M_0$.

\subsection{Structural Reformulation of the Endless Chase}
\textbf{Definition D.3.1 (Permanent Stagnation).} \\
Permanent stagnation occurs if there exists a stage $k_0$ such that for all later stages $k \ge k_0$, the certification window contains no admissible twin index.

\textbf{Proposition D.3.2 (Covering Reformulation).} \\
If permanent stagnation occurs after stage $k_0$, then the family of residue deletions induced by primes $q > \max(B_{k_0})$ forms a covering of the residue set $R_0$ modulo $M_0$.

\textit{Proof (sketch).} \\
Permanent stagnation means that for all sufficiently large $y$, every residue class in $R_0$ is eliminated by at least one congruence condition modulo some prime $q \le y$. Equivalently,
\[
R_0 \subseteq \bigcup_{q > \max(B_{k_0})} E_q,
\]
where each $E_q$ is a union of at most two residue classes modulo $q$, lifted to modulus $M_0$. $\square$

This reformulation is exact and does not rely on probabilistic language or density heuristics.

\subsection{Growth Considerations (Properly Interpreted)}
The certification window grows as
\[
W(y) = \Theta(y^2),
\]
while the proportion of residues surviving new congruence exclusions is bounded above by
\[
\prod_{q \le y} \left(1 - \frac{2}{q}\right) = \Theta\left(\frac{1}{(\log y)^2}\right).
\]

These facts \textbf{do not imply existence} of admissible indices in every window. However, they imply:

\begin{quote}
Any covering of $R_0$ sufficient to enforce permanent stagnation must persist across arbitrarily large scales while compensating for quadratic window growth.
\end{quote}

This places strong structural constraints on such coverings.

\subsection{What Is (and Is Not) Shown}
We therefore conclude:
\begin{itemize}
    \item Permanent stagnation $\Rightarrow$ existence of a covering system relative to $M_0$
    \item The obstruction is \textbf{structural}, not parity-based
    \item CPNF shifts the difficulty from analytic parity to combinatorial covering
\end{itemize}

No claim is made that such coverings are impossible --- only that they are the \emph{only} remaining obstruction.

\section{Philosophical Interpretation (Clarified)}\label{app:E}
\subsection{Anti-Reification Without Anthropomorphism}
The CPNF framework does \textbf{not} attribute intention, randomness, or agency to primes. Instead, it emphasizes:
\begin{itemize}
    \item \textbf{Process dependence}: objects arise stage by stage
    \item \textbf{Irreversibility}: certifications cannot be undone
    \item \textbf{Local justification}: each step depends only on prior certified data
\end{itemize}

The concern with covering systems arises naturally once one refrains from reifying the infinite set of all primes as a completed object.

\subsection{Relation to Process Philosophy}
CPNF aligns with process philosophy in a limited, technical sense:
\begin{itemize}
    \item Mathematical existence is treated as \textbf{procedural emergence}
    \item Obstructions are \textbf{process-relative}, not absolute
    \item Global coordination is replaced by local certification
\end{itemize}

This is consistent with Bergson \cite{bergson1911}, Whitehead \cite{whitehead1929}, and constructive traditions, without importing metaphysical claims about agency.

\section{Limits and Precise Status of the Results}\label{app:F}
\subsection{What Is Proven}
Within CPNF, the paper establishes:
\begin{enumerate}
    \item Certification processes for ordinary primes never terminate.
    \item Twin certification cannot stagnate \textbf{unless} a covering system emerges relative to some fixed modulus.
    \item The parity problem does not obstruct this analysis, because no factor-count distinctions are made.
\end{enumerate}

\subsection{What Remains Open}
CPNF does \textbf{not} prove:
\begin{itemize}
    \item The non-existence of covering systems
    \item Any classical prime distribution result
    \item The Twin Prime Conjecture in its standard form
\end{itemize}

The framework instead \textbf{isolates the exact nature of the remaining obstruction}.

\subsection{Philosophical Contribution (Precisely Stated)}
The contribution of CPNF is methodological and ontological:
\begin{itemize}
    \item It shows how reification introduces analytic obstacles
    \item It demonstrates how process-relative reasoning reframes infinitude
    \item It identifies covering, not parity, as the relevant obstruction
\end{itemize}

This reframing is the central philosophical result of the paper.

\section{Historical and Philosophical Context}\label{app:G}
\subsection{Process Philosophy and Anti-Reification}
\begin{itemize}
    \item \textbf{Henri Bergson (1911)}: Critiqued the spatialization of time and the reification of processes \cite{bergson1911}. CPNF avoids treating the infinite set of primes as a completed object.
    \item \textbf{Alfred North Whitehead (1929)}: Developed process philosophy, emphasizing that reality consists of processes rather than substances \cite{whitehead1929}. CPNF treats primes as certifications in an ongoing process.
    \item \textbf{Gilles Deleuze (1968)}: Extended Bergson's ideas, emphasizing difference and repetition in processes---reflected in CPNF's repetitive certification stages \cite{deleuze1968}.
\end{itemize}

\subsection{Contemporary Philosophy of Mathematics}
\begin{itemize}
    \item \textbf{Øystein Linnebo (2013)}: Modal approaches to the infinite treat infinite sets as potential rather than actual---aligned with CPNF's procedural ontology \cite{linnebo2013}.
    \item \textbf{James Ladyman (2014)}: Structural realism focuses on relations and processes rather than objects \cite{ladyman2014}.
    \item \textbf{Otávio Bueno (2009)}: Emphasizes the representational role of mathematics---CPNF offers an alternative representation of primality \cite{bueno2009}.
    \item \textbf{John Burgess (2015)}: Examines the reliability of mathematical practices---CPNF highlights how different practices yield different obstacles \cite{burgess2015}.
\end{itemize}

\subsection{Constructive Mathematics}
\begin{itemize}
    \item \textbf{L.E.J. Brouwer (1907)}: Intuitionism treats mathematics as mental construction---CPNF's certifications are constructive verifications \cite{brouwer1907}.
    \item \textbf{Per Martin-Löf (1984)}: Type theory provides a formal foundation for constructive mathematics; CPNF could potentially be formalized in such systems \cite{martinlof1984}.
\end{itemize}

\section{Limitations and Future Work}\label{app:H}
\subsection{Formal Verification Needs}
While CPNF arguments are mathematically sound within their framework, formal verification would require:
\begin{enumerate}
    \item \textbf{Complete formalization} in a constructive proof assistant (e.g., Agda, Coq).
    \item \textbf{Explicit bounds} for all asymptotic claims, with error terms.
    \item \textbf{Verification} of the uniformity assumption in fixed-modulus thinning.
\end{enumerate}

\subsection{Potential Extensions}
The CPNF framework could be extended to:
\begin{enumerate}
    \item \textbf{Other prime patterns} (e.g., prime quadruplets, Cunningham chains).
    \item \textbf{Higher-dimensional sieves} for prime k-tuples.
    \item \textbf{Analogous frameworks} in other areas of constructive mathematics (e.g., algebraic number theory).
\end{enumerate}

\subsection{Open Philosophical Questions}
\begin{enumerate}
    \item Can CPNF be formalized in constructive type theory? What would this reveal about the relationship between process ontology and formal systems?
    \item How does CPNF relate to fictionalism about mathematical objects (Bueno, 2009)?
    \item Could process-oriented ontologies help resolve other foundational debates in philosophy of mathematics?
\end{enumerate}

\begin{thebibliography}{99}
\bibitem{bergson1911} Bergson, H. (1911). \emph{Creative Evolution}. Henry Holt and Company.
\bibitem{whitehead1929} Whitehead, A. N. (1929). \emph{Process and Reality}. Macmillan.
\bibitem{brouwer1907} Brouwer, L. E. J. (1907). \emph{Over de grondslagen der wiskunde} (On the Foundations of Mathematics). Amsterdam: Maas \& van Suchtelen.
\bibitem{gauss1801} Gauss, C. F. (1801). \emph{Disquisitiones Arithmeticae}. Leipzig.
\bibitem{linnebo2013} Linnebo, Ø. (2013). The Potential Hierarchy of Sets. \emph{The Review of Symbolic Logic}, 6(2), 205–228.
\bibitem{ladyman2014} Ladyman, J. (2014). Structural Realism and the Relationship between the Special Sciences and Physics. \emph{Synthese}, 191(6), 1097–1114.
\bibitem{bueno2009} Bueno, O. (2009). Mathematical Fictionalism. In \emph{New Waves in Philosophy of Mathematics} (pp. 59–79). Palgrave Macmillan.
\bibitem{burgess2015} Burgess, J. (2015). \emph{Rigor and Structure}. Oxford University Press.
\bibitem{martinlof1984} Martin-Löf, P. (1984). \emph{Intuitionistic Type Theory}. Bibliopolis.
\bibitem{euclid} Euclid. \emph{Elements}. c. 300 BCE.
\bibitem{hardy1923} Hardy, G. H., \& Littlewood, J. E. (1923). Some problems of 'Partitio numerorum'; III: On the expression of a number as a sum of primes. \emph{Acta Mathematica}, 44, 1–70.
\bibitem{mertens1874} Mertens, F. (1874). Ein Beitrag zur analytischen Zahlentheorie. \emph{Journal für die reine und angewandte Mathematik}, 78, 46–62.
\bibitem{selberg1949} Selberg, A. (1949). An elementary proof of the prime-number theorem. \emph{Annals of Mathematics}, 50(2), 305–313.
\bibitem{shapiro1997} Shapiro, S. (1997). \emph{Philosophy of Mathematics: Structure and Ontology}. Oxford University Press.
\bibitem{mancosu2008} Mancosu, P. (2008). \emph{The Philosophy of Mathematical Practice}. Oxford University Press.
\bibitem{ferreiros2016} Ferreirós, J. (2016). \emph{Mathematical Knowledge and the Interplay of Practices}. Princeton University Press.
\bibitem{deleuze1968} Deleuze, G. (1968). \emph{Difference and Repetition}. Presses Universitaires de France.
\end{thebibliography}

\end{document}